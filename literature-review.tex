\documentclass[12pt]{article}
\usepackage{graphicx} % Required for inserting images
\usepackage{titling} % For custom title page
\renewcommand{\familydefault}{\sfdefault}


\title{Analysing the Effectiveness of Consumer-ready Simulation Software in Training for Maritime Navigation Under Pressure}
\author{Richard Lay-Flurrie}
\date{\today}

\pretitle{
  \begin{center}
  \includegraphics[width=0.2\textwidth]{images/logo.jpeg}\vspace{2em} \\ % Add your logo here
  \LARGE
}
\posttitle{
  \end{center}
}

\begin{document}

\maketitle
\thispagestyle{empty} % Remove page number from title page

\begin{center}
  \vfill
  \large
  A thesis submitted for the degree of\\
  Doctor of Philosophy\\
  \vfill
  \normalsize
  School of Computer Science and Electronic Engineering\\
  University of Essex\\
  \vfill
\end{center}

\newpage

\section*{Acknowledgements}

Lorem ipsum dolor sit amet, consectetur adipiscing elit, sed do eiusmod tempor incididunt ut labore et dolore magna aliqua. Ut enim ad minim veniam, quis nostrud exercitation ullamco laboris nisi ut aliquip ex ea commodo consequat. Duis aute irure dolor in reprehenderit in voluptate velit esse cillum dolore eu fugiat nulla pariatur. Excepteur sint occaecat cupidatat non proident, sunt in culpa qui officia deserunt mollit anim id est laborum.

Lorem ipsum dolor sit amet, consectetur adipiscing elit, sed do eiusmod tempor incididunt ut labore et dolore magna aliqua. Ut enim ad minim veniam, quis nostrud exercitation ullamco laboris nisi ut aliquip ex ea commodo consequat. Duis aute irure dolor in reprehenderit in voluptate velit esse cillum dolore eu fugiat nulla pariatur. Excepteur sint occaecat cupidatat non proident, sunt in culpa qui officia deserunt mollit anim id est laborum.

\newpage

\section*{Abstract}

Lorem ipsum dolor sit amet, consectetur adipiscing elit, sed do eiusmod tempor incididunt ut labore et dolore magna aliqua. Ut enim ad minim veniam, quis nostrud exercitation ullamco laboris nisi ut aliquip ex ea commodo consequat. Duis aute irure dolor in reprehenderit in voluptate velit esse cillum dolore eu fugiat nulla pariatur. Excepteur sint occaecat cupidatat non proident, sunt in culpa qui officia deserunt mollit anim id est laborum.

Lorem ipsum dolor sit amet, consectetur adipiscing elit, sed do eiusmod tempor incididunt ut labore et dolore magna aliqua. Ut enim ad minim veniam, quis nostrud exercitation ullamco laboris nisi ut aliquip ex ea commodo consequat. Duis aute irure dolor in reprehenderit in voluptate velit esse cillum dolore eu fugiat nulla pariatur. Excepteur sint occaecat cupidatat non proident, sunt in culpa qui officia deserunt mollit anim id est laborum.

Lorem ipsum dolor sit amet, consectetur adipiscing elit, sed do eiusmod tempor incididunt ut labore et dolore magna aliqua. Ut enim ad minim veniam, quis nostrud exercitation ullamco laboris nisi ut aliquip ex ea commodo consequat. Duis aute irure dolor in reprehenderit in voluptate velit esse cillum dolore eu fugiat nulla pariatur. Excepteur sint occaecat cupidatat non proident, sunt in culpa qui officia deserunt mollit anim id est laborum.

Lorem ipsum dolor sit amet, consectetur adipiscing elit, sed do eiusmod tempor incididunt ut labore et dolore magna aliqua. Ut enim ad minim veniam, quis nostrud exercitation ullamco laboris nisi ut aliquip ex ea commodo consequat. Duis aute irure dolor in reprehenderit in voluptate velit esse cillum dolore eu fugiat nulla pariatur. Excepteur sint occaecat cupidatat non proident, sunt in culpa qui officia deserunt mollit anim id est laborum.

\newpage

\tableofcontents

\newpage

\listoftables

\newpage

\listoffigures

\section{Introduction}

\subsection{The Benefits of Simulation in Training}

\subsection{The Importance of Training for High-pressure Scenarios}

\subsection{Maritime Navigation Under Pressure} 

\section{Literature Review}

\subsection{Maritime Navigation Training}

\subsection{High-pressure Scenarios and Decision-making}

A high-pressure situation is one which can be defined as a scenario where an invidual has a difficult task or decision to make, which is likely to make the individual feel stressed or anxious. In such situations, the individual may experience a range of physiological and psychological responses, such as increased heart rate, sweating, and impaired cognitive function. These responses can impact the individual's ability to think clearly and make effective decisions, which can have serious consequences in critical situations.

Some examples of high-pressure situations include emergencies, meeting deadlines, public speaking and competitive activities. They do not necessarily have to be life-threatening or pose a great risk to the individual to be considered high-pressure. One such example in popular media is Richie's Plank Experience \cite{richiesplankexperience}, a virtual reality game in which the player has to walk across a plank suspended high above the ground. Although the player is not in any real danger, the immersive nature of the virtual reality experience can trigger a fear response, making it a high-pressure scenario.


\cite{el2023walk}







\subsection{Virtual Environments for Training}

\subsection{Measuring Stress Levels}

\subsection{Application of Virtual Environments in Training for High-pressure Scenarios}

\section{Virtual Environments for Training Built in Unreal Engine 5}

\subsection{Introduction}

\subsection{Digital Twins}

\subsection{Unreal Engine 5}

\section{Human Perception and Differentiation of Real and Virtual Environments}

\subsection{Introduction}

\subsection{Methodology}

\subsection{Experiment 1 - Mutilus}

\subsubsection{What is Mutilus?}

\subsubsection{Prototype Experiments}

\subsubsection{Results}

\section{Applications of this Research}

The findings from this research have significant implications for the field of maritime navigation training. By demonstrating the effectiveness of consumer-ready simulation software, this study paves the way for more accessible and cost-effective training solutions. Maritime academies and training centers can integrate these simulations into their curricula, providing students with realistic and immersive training experiences without the need for expensive, specialized equipment.

Furthermore, the research highlights the potential for using virtual environments to train individuals for high-pressure scenarios. This can be extended beyond maritime navigation to other fields such as aviation, military, and emergency response, where decision-making under stress is critical. The ability to create customizable and repeatable training scenarios allows for targeted skill development and assessment.

Additionally, the use of digital twins and advanced simulation technologies, as explored in this study, can be applied to various industries for training, planning, and operational purposes. For instance, in the oil and gas industry, virtual simulations can be used to train personnel on complex procedures and emergency responses, thereby enhancing safety and efficiency.

Overall, this research contributes to the growing body of knowledge on the use of virtual simulations in training and underscores the potential for these technologies to revolutionize training methodologies across multiple domains. 

\subsection{Training for Stressful Scenarios}

\section{Methodology}

\section{Results}

\section{Discussion}

\section{Conclusion}

\section{References}

\bibliographystyle{ieeetrans}
\bibliography{literaturereviewbib}

\section{Appendices}



\end{document}
